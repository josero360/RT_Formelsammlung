\documentclass[10pt,a4paper]{article}
\usepackage[utf8]{inputenc}
\usepackage[ngerman]{babel}
\usepackage[T1]{fontenc}
\usepackage{amsmath}
\usepackage{amsfonts}
\usepackage{amssymb}
\usepackage{graphicx}
\usepackage{lmodern}
\usepackage{physics}
\usepackage[left=1cm,right=1cm,top=2cm,bottom=1.5cm]{geometry}
\usepackage{siunitx}
\usepackage{fancyhdr}
\usepackage{enumerate}
\usepackage{mhchem}
\usepackage{mathtools}
\usepackage{graphicx}
\graphicspath{./Figures}
\usepackage{float}
\usepackage{xcolor}
\usepackage{mdframed}
\usepackage{csquotes}
\usepackage{trfsigns}
\usepackage{capt-of}
\usepackage{listings}

\sisetup{locale=DE}
\sisetup{per-mode = symbol-or-fraction}
\sisetup{separate-uncertainty=true}
\DeclareSIUnit\year{a}
\DeclareSIUnit\clight{c}
\mdfdefinestyle{exercise}{
	backgroundcolor=black!10,roundcorner=8pt,hidealllines=true,nobreak
}

\begin{document}
\twocolumn
\pagestyle{fancy}
\lhead{Regelungstechnik \\ Formelsammlung}
\rhead{\today \\ Maximilian, Binninger}
\section{Grundlagen}
  \subsection{Eigenschaften}
  Eigenschaften LTI-Systeme
  \begin{mdframed}[style=exercise]
    \begin{enumerate}
      \item Stabilität\\
      $\abs{x(t)} < M < \infty \Rightarrow \abs{y(t)} < N < \infty$
      \item Linearität\\
      $W\qty{\sum_{k=1}^{N}a_n x_n(t)}=\sum_{n=1}^{N}W\qty{a_n x_n(t)}$
      \item Zeitinvarianz\\
      $W\qty{x(t-t_0)} =y(t-t_0)$
      \item Kausalität\\
      $t < 0 \Rightarrow x(t)=0 \land y(t)=0$
    \end{enumerate}
  \end{mdframed}
  \subsection{Systemantwort}
  Die Sprung-/Impulsantwort beschreibt Systemantwort vollständig
  \begin{mdframed}[style=exercise]
    \begin{align}
      y(t) &= \int^{\infty}_{-\infty} a(t-\tau) x'(\tau) \dd{\tau}\\
      a(t-\tau) &= W\qty{s(t-\tau)}\nonumber
    \end{align}
  \end{mdframed}
  \begin{mdframed}[style=exercise]
    \begin{align}
      y(t) &= \int_{-\infty}^{\infty} h(t-\tau) x(\tau) \dd{\tau}\\
      h(t-\tau) &= W\qty{\delta(t-\tau)}\nonumber
    \end{align}
  \end{mdframed}
  \subsection{Abtasttheorem}
  Durch die Abtastung wird das Spektrum von $f(t)$ unendlich oft um die Frequenzen $n\cdot \omega_a$ reproduziert.
  \begin{mdframed}[style=exercise]
    \begin{align}
      F_A(\omega) &= \frac{1}{T_A} \sum_{n=-\infty}^{\infty} F(\omega-n\omega_A)\\
      2\omega_g &\leq \omega_A\nonumber
    \end{align}
  \end{mdframed}
  \section{Systemtechnik}
  \subsection{Modellbildung}
  Hinweise zum aufstellen der Differentialgleichung eines Systems:
  \begin{mdframed}[style=exercise]
  \begin{enumerate}
      \item Bestimmmung der Ein- und Ausgangsgrößen\\
      \item Suche nach dem beschreibenden Gleichgewicht\\
      \item In der Gleichung dürfen nur Konstanten, sowie die Ein- und Augangsgrößen in beliebiger Ableitung vorkommen\\
      \item Andere Variablen müssen durch erlaubte Größen ersetzt werden (Dazu können i.a. physikalische Gleichungen benutzt werden)

    \end{enumerate}
\end{mdframed}
\subsection{Signalflussplan/Blockschaltbild}

\section{Grundlagen}
  \subsection{Eigenschaften}
  Eigenschaften LTI-Systeme
  \begin{mdframed}[style=exercise]
    \begin{enumerate}
      \item Stabilität\\
      $\abs{x(t)} < M < \infty \Rightarrow \abs{y(t)} < N < \infty$
      \item Linearität\\
      $W\qty{\sum_{k=1}^{N}a_n x_n(t)}=\sum_{n=1}^{N}W\qty{a_n x_n(t)}$
      \item Zeitinvarianz\\
      $W\qty{x(t-t_0)} =y(t-t_0)$
      \item Kausalität\\
      $t < 0 \Rightarrow x(t)=0 \land y(t)=0$
    \end{enumerate}
  \end{mdframed}
  \subsection{Systemantwort}
  Die Sprung-/Impulsantwort beschreibt Systemantwort vollständig
  \begin{mdframed}[style=exercise]
    \begin{align}
      y(t) &= \int^{\infty}_{-\infty} a(t-\tau) x'(\tau) \dd{\tau}\\
      a(t-\tau) &= W\qty{s(t-\tau)}\nonumber
    \end{align}
  \end{mdframed}
  \begin{mdframed}[style=exercise]
    \begin{align}
      y(t) &= \int_{-\infty}^{\infty} h(t-\tau) x(\tau) \dd{\tau}\\
      h(t-\tau) &= W\qty{\delta(t-\tau)}\nonumber
    \end{align}
  \end{mdframed}
  \subsection{Abtasttheorem}
  Durch die Abtastung wird das Spektrum von $f(t)$ unendlich oft um die Frequenzen $n\cdot \omega_a$ reproduziert.
  \begin{mdframed}[style=exercise]
    \begin{align}
      F_A(\omega) &= \frac{1}{T_A} \sum_{n=-\infty}^{\infty} F(\omega-n\omega_A)\\
      2\omega_g &\leq \omega_A\nonumber
    \end{align}
  \end{mdframed}
\section{Grundlagen}
  \subsection{Eigenschaften}
  Eigenschaften LTI-Systeme
  \begin{mdframed}[style=exercise]
    \begin{enumerate}
      \item Stabilität\\
      $\abs{x(t)} < M < \infty \Rightarrow \abs{y(t)} < N < \infty$
      \item Linearität\\
      $W\qty{\sum_{k=1}^{N}a_n x_n(t)}=\sum_{n=1}^{N}W\qty{a_n x_n(t)}$
      \item Zeitinvarianz\\
      $W\qty{x(t-t_0)} =y(t-t_0)$
      \item Kausalität\\
      $t < 0 \Rightarrow x(t)=0 \land y(t)=0$
    \end{enumerate}
  \end{mdframed}
  \subsection{Systemantwort}
  Die Sprung-/Impulsantwort beschreibt Systemantwort vollständig
  \begin{mdframed}[style=exercise]
    \begin{align}
      y(t) &= \int^{\infty}_{-\infty} a(t-\tau) x'(\tau) \dd{\tau}\\
      a(t-\tau) &= W\qty{s(t-\tau)}\nonumber
    \end{align}
  \end{mdframed}
  \begin{mdframed}[style=exercise]
    \begin{align}
      y(t) &= \int_{-\infty}^{\infty} h(t-\tau) x(\tau) \dd{\tau}\\
      h(t-\tau) &= W\qty{\delta(t-\tau)}\nonumber
    \end{align}
  \end{mdframed}
  \subsection{Abtasttheorem}
  Durch die Abtastung wird das Spektrum von $f(t)$ unendlich oft um die Frequenzen $n\cdot \omega_a$ reproduziert.
  \begin{mdframed}[style=exercise]
    \begin{align}
      F_A(\omega) &= \frac{1}{T_A} \sum_{n=-\infty}^{\infty} F(\omega-n\omega_A)\\
      2\omega_g &\leq \omega_A\nonumber
    \end{align}
  \end{mdframed}
  \section{Zusammenwirken mehrerer Systeme}
  \subsection{Fourierreihe}
  \begin{mdframed}[style=exercise]
    \begin{align}
      f(t) &= \sum_{n=0}^{\infty} \qty[a_n \cos(n \omega_0 t) + b_n \sin(n \omega_0 t)]\\
      a_n &= \frac{2}{T}\int_{-T/2}^{T/2}f(t)\cos(n\omega_0 t)\dd{t}\nonumber\\
      b_n &= \frac{2}{T}\int_{-T/2}^{T/2}f(t)\sin(n\omega_0 t)\dd{t}\nonumber
    \end{align}
  \end{mdframed}
  \pagebreak
  \subsection{Fourierreihe, komplex}
  \begin{mdframed}[style=exercise]
    \begin{align}
      f(t) &= \sum_{n=-\infty}^{\infty} c_n e^{jn\omega_0 t}\\
      c_n &= \frac{1}{T} \int_{-T/2}^{T/2} f(t) e^{-jn\omega_0 t}\dd{t}\nonumber
    \end{align}
  \end{mdframed}
  \subsection{Fourierintegral}
  \begin{mdframed}[style=exercise]
    \begin{align}
      f(t) &= \frac{1}{2\pi} \int_{-\infty}^{\infty} F(\omega) e^{j\omega t} \dd{\omega}\\
      F(\omega) &= \int_{-\infty}^{\infty} f(t) e^{-j\omega t} \dd{t}
    \end{align}
  \end{mdframed}
  \subsubsection{Eigenschaften}
  \begin{mdframed}[style=exercise]
    \begin{enumerate}
      \item Linearität\\
      $a f_1(t) + b f_2(t) \laplace a F_1(\omega) + b F_2(\omega)$
      \item Zeitverschiebung\\
      $f(t-t_0) \laplace F(\omega)e^{-j\omega t_0}$
      \item Frequenzverschiebung\\
      $f(t) e^{\pm j\omega_0 t}\laplace F(\omega\mp \omega_0)$
      \item Faltung\\
      $f_1(t)*f_2(t)\laplace F_1(\omega)\cdot F_2(\omega)$\\
      $f_1(\omega)\cdot f_2(\omega)\laplace \frac{1}{2\pi}F_1(t)*F_2(t)$
    \end{enumerate}
  \end{mdframed}
  \subsection{DFT}
  \begin{mdframed}[style=exercise]
    \begin{align}
      x_n &= \frac{1}{N} \sum_{k=0}^{N-1} X_k\cdot e^{i 2 \pi k n / N}\\
      X_k &= \sum_{n=0}^{N-1} x_n\cdot e^{-i 2 \pi k n / N}
    \end{align}
  \end{mdframed}
  \subsubsection{FFT}
  \begin{center}
    \includegraphics[width=.35\textwidth]{Figures/butterfly}
    \captionof{figure}{FFT}
  \end{center}
  \pagebreak
  \subsection{Hilbert Transformation}
  \begin{mdframed}[style=exercise]
    \begin{align}
      x_{\mathrm{ht}}(t) &= x_{\mathrm{r}}(t) * h(t)\\
      H(\omega) &= -j \, \text{sgn}(\omega)
    \end{align}
  \end{mdframed}
  \subsection{z Transformation}
  \begin{mdframed}[style=exercise]
    \begin{align}
      X(z) &= \sum_{n=-\infty}^{\infty} x(n) z^{-n}\\
      x(n) &= \frac{1}{2\pi j} \oint_c X(Z) z^{n-1}\dd{z}
    \end{align}
  \end{mdframed}
  \subsubsection{Übertragungsfunktion}
  \begin{mdframed}[style=exercise]
    \begin{align}
      H(Z) &=\frac{Y(Z)}{X(z)}=\frac{\sum_{k=0}^q b_k z^{-k}}{\sum_{k=0}^p a_k z^{-k}}=k\frac{\prod_{k=1}^q (1-z_k z^{-1})}{\prod_{k=1}^p (1-p_k z^{-1})}
    \end{align}
  \end{mdframed}
  \subsubsection{Verschiebung im Zeitbereich}
  \begin{mdframed}[style=exercise]
    \begin{align}
      Y(z) &= \sum_{n=0}^{\infty} \qty[x(n-m)] z^{-n} =  z^{-m} X(z)\\
      Y(z) &= \sum_{n=0}^{\infty}\qty[x(n+m)]z^{-n} = z^{m}\qty[x(t)-\sum_{n=0}^{m-1} x(n) z^{-n}]
    \end{align}
  \end{mdframed}
  \section{Digitale Regler}
  \subsection{FIR}
  \begin{mdframed}[style=exercise]
    \begin{align}
      y[n] &= \sum_{k=0}^{q} b_k x(n-k)
    \end{align}
  \end{mdframed}
  \begin{center}
    \includegraphics[width=.4\textwidth]{Figures/fir.png}
  \end{center}
  \subsection{IIR}
  \begin{mdframed}[style=exercise]
    \begin{align}
      y[n] &= \sum_{k=0}^{q} b_k x(n-k) - \sum_{k=1}^{p} a_k y(n-k)
    \end{align}
  \end{mdframed}
  \begin{center}
    \includegraphics[width=.35\textwidth]{Figures/df1.png}
    \captionof{figure}{Direkt Form 1}
  \end{center}
  \begin{center}
    \includegraphics[width=.35\textwidth]{Figures/df2.png}
    \captionof{figure}{Direkt Form 2}
  \end{center}
  \section{Systembeschreibung im Zustandsraum}
  \subsection{Allgemein (Mehrgrößensystem MIMO) }
  \begin{mdframed}[style=exercise]
    \begin{equation}
      \dot{\Vec{x}}(t) = A\vec{x}(t) + B\vec{u}(t); \quad x(0) = x_{0}; \\*
      \vec{y}(t) = C\vec{x}(t) + D\vec{u}(t)
    \end{equation}
  \end{mdframed}
  \begin{center}
    \includegraphics[width=.5\textwidth]{Figures/Signalflussplan.png}
    \captionof{figure}{Signalflussplan}
  \end{center}
  \begin{mdframed}[style=exercise]
    \begin{align*}
      \dot{x}_{1} &= 0*x_{1} +0*x_{2} +0*x_{3} +1*u_{1}+0*u_{2} \\
      \dot{x}_{2} &= K*x_{1} +0*x_{2} +0*x_{3} +0*u_{1}+0*u_{2} \\
      \dot{x}_{3} &= 0*x_{1} +0*x_{2} +0*x_{3} +H*J*u_{1}+J*u_{2} \\
      \dot{y}_{1} &= 0*x_{1} +1+x_{2} +0*x_{3} +0*u_{1}+0*u_{2} \\
      \dot{y}_{2} &= 0*x_{1} +0*x_{2} +l*x_{3} +0*u_{1}+0*u_{2} \\
    \end{align*}
  \end{mdframed}

  \begin{mdframed}[style=exercise]
    % \begin{align*}
    % A &= \left[\begin{array}{rrr}
    % 0&0&0\\
    % K&0&0\\
    % 0&0&0\\
    % \end{array}\right] \
    % B &= \left[\begin{array}{rr}
    % 1&0\\
    % 0&0\\
    % H*J&J\\
    % \end{array}\right]

    % C &=\left[\begin{array}{rrr}
    % 0&1&0\\
    % 0&0&l\\
    % \end{array}\right] \

    % D &=\left[\begin{array}{rrr}
    % 0&0\\
    % 0&0\\
    % \end{array}\right]

    % \end{align*}
    \[
        \ A = \begin{bmatrix}
            0&0&0\\
            K&0&0\\
            0&0&0
        \end{bmatrix}
        C = \begin{bmatrix}
            0&1&0\\
            0&0&l\\
        \end{bmatrix}\]\\
        \[B = \begin{bmatrix}
            1&0\\
            0&0\\
            H*J&J
        \end{bmatrix} \
        D = \begin{bmatrix}
            0&0\\
            0&0
        \end{bmatrix}\]

  \end{mdframed}
  \subsection{Programmtechnische Umsetzung}
Zaehler und Nenner der z-Uebertragungsfunktion durch die hoechste Potenz teilen\\
  \begin{mdframed}[style=exercise]
    \begin{lstlisting}
      while(1){
      waitinterrupt();
      xout2 = xout1;
      xout1 = xout;
      xin2 = xin1;
      xin1 = xin;
      input(xin);
      xa = k*xout2 - j*xin1 + o*xout1;
      output(xa);
      }
    \end{lstlisting}
  \end{mdframed}
\end{document}
