\section{Grundlagen}
Linearisierung um Arbeitspunkt:
\begin{align*}
	x_{a}(t)=x_{a,AP}+\Delta x_{a}(t) \approx x_{a,AP}+\sum\left(\frac{\partial f}
	{\partial x_{e,Ap}} \cdot \Delta x_{e}(t)\right)
\end{align*}
Kräftegleichungen:
\begin{mdframed}[style=exercise]
	Federkraft: $F_F$ = $k_F \cdot x$\\
	Dampfkraft: $F_D$ = $k_D \cdot v$ = $k_D \cdot \dot{x}$\\
	Trägheitskraft: $F_{Tr}$ = $m\cdot a$ = $m\cdot \ddot{x}$\\
	Erdanziehungskraft: $F_G$ = $m\cdot g$
\end{mdframed}
Moment Gleichungen:
\begin{mdframed}[style=exercise]
	Widerstandsmoment: $M_w$ = $k_D \cdot \omega$\\
	Trägheitsmoment: $M_{TR}$ = J$\cdot \dot{\omega}$
\end{mdframed}
Spannungsgleichung:
\begin{mdframed}[style=exercise]
	\[
		U = L\cdot \frac{di}{dt}+i\cdot R+\frac{1}{C} \cdot \int i
	\]
\end{mdframed}
Für kleine Winkel $\alpha$ gilt: sin($\alpha$) = $\alpha$\\
Rotation in Flüssigkeit:
\[
	M=M_{\texttt{Träg}}+M_{\texttt{Brems}}=J \cdot \dot{\omega} +k_{\texttt{Flüssigkeit}} \cdot \omega
\]\\
Partialbruchzerlegung (siehe Papula s.157ff)\\
\\
Anfangswertsatz: $x(t \rightarrow +0)=\lim _{s \rightarrow \infty} s \cdot X(s)$\\
Endwertsatz: $x(t \rightarrow \infty)=\lim _{s \rightarrow 0} s \cdot X(s)$