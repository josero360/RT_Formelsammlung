\section{Grundlagen}
Linearisierung um Arbeitspunkt:
\begin{align*}
	x_{a}(t) &= x_{a,AP}+\Delta x_{a}(t) \approx x_{a,AP}+\sum\left(\frac{\partial f}{\partial x_{e,Ap}} \cdot \Delta x_{e}(t)\right)\\
	f(x) &= y_0 + k \cdot (x - x_0) = y_0 + \left(\frac{y_2 - y_1}{x_2 - x_1}\right) \cdot (x - x_0)
\end{align*}

\begin{mdframed}[style=exercise]
	Anfangswertsatz: $x(t \rightarrow +0)=\lim _{s \rightarrow \infty} s \cdot X(s)$\\
	Endwertsatz: $x(t \rightarrow \infty)=\lim _{s \rightarrow 0} s \cdot X(s)$
\end{mdframed}

Kräftegleichungen:
\begin{mdframed}[style=exercise]
	Federkraft: $F_F$ = $k_F \cdot x$\\
	Dampfkraft: $F_D$ = $k_D \cdot v$ = $k_D \cdot \dot{x}$\\
	Trägheitskraft: $F_{Tr}$ = $m\cdot a$ = $m\cdot \ddot{x}$\\
	Erdanziehungskraft: $F_G$ = $m\cdot g$
\end{mdframed}

Momentgleichungen:
\begin{mdframed}[style=exercise]
	Widerstandsmoment: $M_w$ = $k_D \cdot \omega$\\
	Trägheitsmoment: $M_{TR}$ = J$\cdot \dot{\omega}$
\end{mdframed}

Spannungsgleichung:
\begin{align*}
	U &= L\cdot \frac{di}{dt}+i\cdot R+\frac{1}{C} \cdot \int i	\\
	U &= L \cdot \dot{I} \\
	I &= C \cdot \dot{U} \\
\end{align*}

Beispiel:
\begin{align*}
	M = M_{TR}+M_w &= J \cdot \dot{\omega} +k_D \cdot \omega \\
	\dot{\omega} &= \frac{M - k_D \cdot \omega}{J}
\end{align*}