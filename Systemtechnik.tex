\section{Systemtechnik}
\subsection{Modellbildung}
Hinweise zum aufstellen der Differentialgleichung eines Systems:
\begin{mdframed}[style=exercise]
	\begin{enumerate}
		\item Bestimmung der Ein- und Ausgangsgrößen
		\item Suche nach dem beschreibenden Gleichgewicht
		\item In der Gleichung dürfen nur Konstanten, sowie die Ein- und
		      Ausgangsgrößen in beliebiger Ableitung vorkommen
		\item Andere Variablen müssen durch erlaubte Größen ersetzt werden\\
		      \footnotesize
		      (Dazu können i.a. physikalische Gleichungen benutzt werden)
	\end{enumerate}
\end{mdframed}

\subsection{Signalflussplan/Blockschaltbild}
Erzeugung des Signalflussplans aus der Zugehörigen DGL.
\begin{mdframed}[style=exercise]
	\begin{enumerate}
		\item für technische Realisierung gilt: m < n;

		\item DGL. nach höchster Ableitung der Ausgangsgröße auflösen

		\item höchste Ableitung der Ausgangsgröße geht auf den Eingang des ersten Integrators\\
		      \footnotesize
		      (Laplace-Trans ersetzt das Integrieren mit einer Division mit „s“)
	\end{enumerate}
\end{mdframed}
\begin{center}
	\includegraphics[width=0.96\columnwidth]{Figures/Signalflussplan12.png}
\end{center}

Erzeugung des Signalflussplans (und Übertragungsfunktion) eines Systems mit der DGL.:
\begin{align*}
	  b_{n} \overset{(n)}{x}_{e}+\ldots+b_{2} \ddot{x}_{e}+b_{1} \dot{x}_{e}+b_{0} x_{e} & \\
	  (---------------& \quad = F(s) )\\
	= a_{n} \overset{(n)}{x}_{a}+\ldots+a_{2} \ddot{x}_{a}+a_{1} \dot{x}_{a}+a_{0} x_{a} &
\end{align*}

Signalflussplan kann allgemein gezeichnet werden: (nötig, falls eine Ableitung von $x_e$ existiert)

\includegraphics[width=0.9\columnwidth]{Figures/SFmitDGL.png}

% \subsection{Stabilität}
% \begin{mdframed}[style=exercise]
% 	BIBO-Stabilität (Bounded Input/ Bounded Output-> begrenzt):
% 	für ein begrenztes $x_e$ gibt es immer ein begrenztes $x_a$
% \end{mdframed}

\subsection{Zusammenhang DGL, Frequenzgang, Übertragungsfkt.}
\begin{align*}
	F_e &= m \ddot{x} + k_D \dot{x} + k_F x \\
	F(j \omega) = \frac{\Sigma x_e}{\Sigma x_a} & \qquad j^2 = -1\\
	F(j \omega) &= \frac{1}{m(j \omega)^2+k_D j \omega+k_F} \\
				&= \frac{1}{-m \omega^2+h_D \omega+h_F} \\
	\text{konj. komp. erweitert}			&= \frac{\left(k_F-m \omega^2\right)-k_D j \omega}{\left(k_F-m \omega^2\right)^2-(-1)\left(k_D \omega\right)^2} \\
				&= \frac{\left(k_F-m \omega^2\right)-k_{D j} \omega}{\left(k_F-m \omega^2\right)^2+\left(k_D \omega\right)^2} \\
	F(s)		&= \frac{1}{m s^2+k_D s+h_F}
\end{align*}

\newpage
\subsection{Ortskurven und Frequenzkennlinien}
Ortskurvendarstellung:\\ 
(In Darstellung $F(j \omega) = a + jb$ eintragen)
\begin{center}
	\includegraphics[width=0.86\columnwidth]{Figures/Ortskurve.png}
\end{center}
% \begin{mdframed}[style=exercise]
% 	Für wachsendes $\omega$ werden die komplexen Werte F(j$\omega$) in die
% 	komplexe F-Ebene eingetragen und zur Ortskurve verbunden.\\ Jeder
% 	Ortskurvenpunkt kann jetzt als Zeiger gedeutet werden.
% \end{mdframed}

Bodediagramm Darstellung:\\ 
(In Darstellung $F(j \omega) = A(rg) \measuredangle \varphi$ eintragen)
\begin{center}
	\includegraphics[width=0.86\columnwidth]{Figures/Bodediagramm.png}
\end{center}
\begin{mdframed}[style=exercise]
	% Der Amplitudengang A($\omega$) wird in doppelt logarithmischer Darstellung
	% aufgetragen,\\ der Phasengang y($\omega$)) halblogarithmisch. Gemeinsame
	% Abszisse ist $\omega$.

	% Bei diesem Beispiel (PT1-Glied) ist deutlich der\\ Tiefpass-Charakter zu
	% erkennen.

	% Verkettete Funktionen im Bodediagramm resultieren als Produkt der
	% Einzelübertragungsfunktionen.

	% D.h. 
	Beispiel PT1-Glied\\
	Verstärkung wird multipliziert und Phasenverschiebung addiert.\\ Das
	heißt: Sowohl Phasengang und Amplitudengang werden graphisch addiert!
\end{mdframed}

% \subsection{$\mathbf{F(s)}$ in Pol- und Nullstellenform}
% \begin{mdframed}[style=exercise]
% 	Zähler- und Nennerpolynom von F(s) besitzt Nullstellen. Diese sind von
% 	$a_v$ und $b_u$ abhängig.\\
% 	Nullstellen des Zählers sind Nullstellen von F(s)\\
% 	Nullstellen des Nenners sind Polstellen von F(s)\\
% 	Wenn Pole $s_pv$ und Nullstellen $s_nu$ bekannt, kann man F(s) mit dem
% 	Faktor Q in faktorisierter Form darstellen.
% \end{mdframed}

% \[
% 	F(s)=Q \cdot \frac{\prod_{\mu=1}^{m}\left(s-s_{N \mu}\right)}{\prod_{v=1}^{n}\left(s-s_{P V}\right)}
% 	\qquad {mit \;} Q = \frac{b_m}{a_n}
% \]

% \begin{mdframed}[style=exercise]
% 	Die Stabilität von $F(s)$ kann anhand der Lage der Pole $s_pv$ in der s-Ebene beurteilt werden.\\ 
% 	stabil, wenn alle Pole $s_{PV}$ in der linken
% 	s-Halbebene liegen.\\ Instabile Pole
% 	NICHT durch Reihenschaltung mit Nullstelle kompensieren!\\
% 	Bedeutung Polstelle:\\
% 	Je weiter links $\Rightarrow$ schnellerer Einschwingvorgang.\\
% 	$\Rightarrow$ Pole weiter links können vernachlässigt werden.
% \end{mdframed}
% \begin{mdframed}[style=exercise]
% 	Bedeutung Nullstelle:\\
% 	bewirken differenzierendes Verhalten (Beschleunigung des Systems)
% 	$\Rightarrow$ NS weiter links können vernachlässigt werden.
% \end{mdframed}