\section{Digitale Regler}
\subsection{Allgemeines}
\subsubsection{z-Transformation}

Wert der bei t = k $\cdot T_A$ ausgegeben wird, wird bei t = (k-1) $\cdot T_A$
eingelesen. (Verzögerung um einen Abtastschritt):

\includegraphics[width=0.9\columnwidth]{Figures/zTrans.png}

Bei der z-Transformation entspricht $e^{-s \cdot TA}$ der
Laplace-Transformation dem Ausdruck $z^{-1}$. Bzw. z $\hat{=}$ $e^{s \cdot
			TA}$\\ Transformation vom s-Bereich in den z-Bereich:
\[
	s \ \hat{=}\  e^{s \cdot TA}
\]

Vorwärts-Differenzen-quotient
\[
	s \ \hat{=} \ \frac{z-1}{T_A}
\]

$\Rightarrow$ Der digitale Regler kann folgend berechnet werden:
\[
	F(z) = F(z)|_{s= \frac {z-1}{T_A}}
\]

Tustinsche Formel
\[
	s \ \hat{=} \ \frac{2}{T_A} \cdot \frac{z-1}{z+1}
\]

Sinnvolle Abtastzeit:\\
kleinste Polstelle $S_P \Rightarrow T_P = \frac{1}{S_P}$
\[
	T_{\textnormal{sinnvoll}} = T_P \cdot \left(\frac{1}{10} \dots \frac{1}{20}\right)
\]